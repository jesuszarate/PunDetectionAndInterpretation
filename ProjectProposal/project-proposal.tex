\documentclass{article}
\documentclass[11pt]{article}
\usepackage[margin=2.54cm]{geometry}
\usepackage[colorlinks,urlcolor=blue]{hyperref}
\usepackage{url}
\usepackage{graphicx}
\usepackage{listings}

\usepackage{color}
\definecolor{gray}{rgb}{0.4,0.4,0.4}
\definecolor{darkblue}{rgb}{0.0,0.0,0.6}
\definecolor{cyan}{rgb}{0.0,0.6,0.6}

\lstset{
language=XML,
morekeywords={encoding,
xs:schema,xs:element,xs:complexType,xs:sequence,xs:attribute}
}
\usepackage{enumitem}

\lstset{
language=XML,
morekeywords={encoding,
xs:schema,xs:element,xs:complexType,xs:sequence,xs:attribute}
}

\setlist[enumerate,1]{start=0} % only outer nesting level

\title{Pun Identification and Extraction}
\author{Team: PUN}
\author{Johnny Le, Jesus Zarate}

\date{\today}

\begin{document}

\maketitle


\section{Task}

In our project we aim to detect and extract puns from a set of punning jokes. Puns
belong to a class of language constructs, where the speaker or writer intends ambiguity
on a certain word, allowing the word to carry two or more separate meanings
simultaneously.
Typically word sense disambiguation \big(WSD\big) assumes that for each
 word there exists a single unambiguous communicative intention.\\

\noindent There are two different types of puns, {\textbf{homographic} } puns and {\textbf{heterographic} } puns.
Homographic or perfect puns are puns where the two meanings share the same pronunciation.
On the other hand heterophonic or imperfect puns are those that rely on similar sounding
signs, where the signs are considered as written rather than spoken sequences




\section{Data}\label{sec:data}

%The data we will be using for this system will be
We will be using the
\href{http://alt.qcri.org/semeval2017/task7/}{SemEval-2017 Task 7}
data.
The data is in XML format with each pun joke listed sequentially.\\

Example of a pun joke:\\

\begin{lstlisting}
    <text id="hom_4">
    <word id="hom_4_1">A</word>
    <word id="hom_4_2">ditch</word>
    <word id="hom_4_3">digger</word>
    <word id="hom_4_4">was</word>
    <word id="hom_4_5">entrenched</word>
    <word id="hom_4_6">in</word>
    <word id="hom_4_7">his</word>
    <word id="hom_4_8">career</word>
    <word id="hom_4_9">.</word>
    </text>
\end{lstlisting}

\noindent There is also a text file that illustrates the answer format for the scoring program that came with the data.
The format of the file consists of two fields separated by horizontal whitespace
(a single tab or space character).
The first field is the ID of a text from the XML field.
The second field is the ID of the one word in that text which is a pun.\\

Example answer for the pun joke above from the text file:\\

hom\_4   hom\_4\_5


\section{Evaluation}\label{sec:evaluation}

Since the data came from \href{http://alt.qcri.org/semeval2017/task7/}{SemEval-2017 Task 7}
which contains a scoring program, and it uses the F-measure statistic.\\

\noindent We will be implementing our evaluation program using python.
We will compute the following:\\


\begin{itemize}

\item [] {\textbf{Recall} }


The number of correct puns $($location of the pun word$)$ extracted divided by the number
of puns in the gold file. \\

\item [] {\textbf{Precision} }\\

The precision will be computed by taking the number of {\textbf{correct} } puns extracted divided
by the total number of puns extracted by our system.

\item [] {\textbf{F-measure} }\\

The F-measure is how we're going to be measuring the performance of our extraction system.

\[F_1 = 2 * \frac{Precision * Recall} {Precision + Recall} \]

\end{itemize}

\noindent The data comes with two sets of data a training data, (which we will do the training with)
and test set. \\

\noindent The training data will be split into two sets, one for training and the rest for test and then tune.
Finally the test data will be used to evaluate the system.



\section{Demo Interface}\label{sec:demoInterface}

The demo interface will consist of running a python script on the command-line.\\

\begin{center}
python extract.py input-file.txt\\
\end{center}


\noindent Executing the command above will yield two files.
The first file identified pun word in
each text, and the second file will be the evaluation results discussed in the previous
section.\\

\begin{center}
subtask1-heterographic-test.results\\
subtask1-heterographic-test.score
\end{center}


\end{document}